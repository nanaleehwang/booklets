\documentclass{article}
\usepackage{amsmath}
\usepackage{amsfonts}
\usepackage{amssymb}

\title{Chapter 1: Vectors and Vector Spaces}
\author{Dr. Jane Smith}

\begin{document}
\maketitle

\section{Introduction to Vectors}

A vector is a mathematical object that has both magnitude and direction. In this chapter, we will explore the fundamental properties of vectors and vector spaces.

\subsection{Definition of a Vector}

A vector can be represented in multiple ways:
\begin{itemize}
\item Geometrically as an arrow in space
\item Algebraically as an ordered list of numbers
\item Abstractly as an element of a vector space
\end{itemize}

\subsection{Vector Operations}

The two fundamental operations on vectors are:
\begin{enumerate}
\item \textbf{Vector Addition}: $(a_1, a_2, \ldots, a_n) + (b_1, b_2, \ldots, b_n) = (a_1 + b_1, a_2 + b_2, \ldots, a_n + b_n)$
\item \textbf{Scalar Multiplication}: $c(a_1, a_2, \ldots, a_n) = (ca_1, ca_2, \ldots, ca_n)$
\end{enumerate}

These operations satisfy several important properties:
\begin{itemize}
\item Commutativity: $\mathbf{u} + \mathbf{v} = \mathbf{v} + \mathbf{u}$
\item Associativity: $(\mathbf{u} + \mathbf{v}) + \mathbf{w} = \mathbf{u} + (\mathbf{v} + \mathbf{w})$
\item Distributivity: $c(\mathbf{u} + \mathbf{v}) = c\mathbf{u} + c\mathbf{v}$
\end{itemize}

\end{document}